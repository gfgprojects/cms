\documentclass{article}
\usepackage{hyperref,color}
\begin{document}

\begin{center}
	{\bf
	Instruction manual for 

An Agent-based model of the economy with consumer credit
}

by

%Paola D'Orazio and Gianfranco Giulioni
Anonymised for review
\end{center}


This document reports the instructions for installing and running the model in Unix like Operating Systems using a command line approach. Therefore, the instructions will be also valid for Linux and recent Mac machines.

The examples and the command line outcomes given below relate to a user named \verb+coolcoder+. You should easily be able to adapt the paths to you own user account. 

The following colors are used:\\
\color{red}red \color{black} to denote a command;\\
\color{blue}blue \color{black} to denote an ordinary file in command line output;\\
\color{green}green \color{black} to denote an executable file in command line output;\\
\color{magenta}magenta \color{black} to denote the contents of text files.\\

\section{Install Repast}

Follow the instruction in
\url{http://repast.sourceforge.net}
to install the newest Repast Simphony release.

\section{Install the model}

Download the model installation bundle \verb+modelJasss.zip+ from \\
\url{https://www.openabm.org/model/4990}

Suppose the model destination folder is called \verb+models+ and has the following absolute path: \\
\verb+/Users/coolcoder/models+

Move \verb+modelJasss.zip+ to this directory and \verb+cd+ into it.

The content of the folder is

\vskip2mm
\noindent\verb+~/models$ +\color{red}\verb+ls+ \color{blue}\\
\verb+modelJass.zip+
\color{black}

\vskip2mm
Extract the zip file:

\vskip2mm
\noindent\verb+~/models$ +\color{red}\verb+unzip modelJasss.zip+ \color{blue}\\
\verb+Archive:  modelJasss.zip+\\
\verb+   creating: modelJasss/+\\
\verb+   creating: modelJasss/bin/+\\
\verb+   creating: modelJasss/bin/modelJasss/+\\
\verb+  inflating: modelJasss/bin/modelJasss/Bank.class+\\ 
\verb+  inflating: modelJasss/bin/modelJasss/Consumer.class+\\ 
\verb+  . . .+\\
\verb+  . . .+\\
\color{black}
\noindent\verb+~/models$ +\color{red}\verb+ls+ \color{blue}\\
\verb+modelJasss+\\
\verb+modelJasss.zip+\\
\color{black}

\verb+cd+ into the \verb+modelJasss+ directory and get its absolute path  



\vskip2mm
\noindent\verb+~/models$ +\color{red}\verb+cd modelJasss+ \color{black}\\
\verb+~/models/modelJasss$ +\color{red}\verb+pwd+ \color{blue}\\
\verb+/Users/coolcoder/models/modelJasss+
\color{black}
\vskip2mm


Save this information because it will be used in the configuration phase. \\ We will refer to it as the model base directory.


\section{Configuration}

Create a new directory outside the model base directory. \\ We will refer to it as the data directory.

Suppose the data directory is called \verb+modelJasssData+ and has the following absolute path: \\
\verb+/Users/coolcoder/Documents/modelJasssData+

\verb+cd+ into the data directory.

Find out the Repast installation directory:

\vskip2mm
\noindent\verb+~/Documents/modelJasssData$ +\color{red}\verb+sudo find / -name "repast.simphony.core*"+ \color{black}
\verb+Password:+ \\ \color{blue}
\verb+/Users/coolcoder/abm_java_libraries/repast/repast.simphony.core_2.3.1 +\\
\color{black}
\vskip2mm

In this expression, \verb+/Users/coolcoder/abm_java_libraries/repast+ is repast base directory and \verb+2.3.1+ is repast version.

Prepare a text file named \verb+paths.txt+ having the repast base directory in its first line, repast version in the second line and the model base directory in the third line:

\color{magenta}
\vskip2mm \noindent
\verb+/Users/coolcoder/abm_java_libraries/repast+ \\
\verb+2.3.1+ \\
\verb+/Users/coolcoder/models/modelJasss+
\vskip2mm

\color{black}
You must adapt the paths and the repast version of this file to your settings.

Save this file into the data directory.

\vskip2mm
Download the \verb+configure.txt+ file from \\
\url{https://www.openabm.org/model/4990}

Move it in the data folder.

The contents of your data folder is now:

\vskip2mm
\noindent\verb+~/Documents/modelJasssData$ +\color{red}\verb+ls+ \color{blue}\\
\verb+configure.txt+ \\
\verb+paths.txt+
\color{black}

\vskip2mm
Rename the \verb+configure+ file, make it executable and run it:

\vskip2mm
\noindent\verb+~/Documents/modelJasssData$ +\color{red}\verb|mv configure.txt configure| \color{black}\\
\noindent\verb+~/Documents/modelJasssData$ +\color{red}\verb|chmod +x configure| \color{black}\\
\verb+~/Documents/modelJasssData$ +\color{red}\verb|./configure| \color{black}

\vskip2mm
This creates three additional files:

\vskip2mm
\noindent\verb+~/Documents/modelJasssData$ +\color{red}\verb+ls+ \color{blue}\\
\verb+compile+\\ \color{green}
\verb+configure+\\ \color{blue}
\verb+paths.txt+ \\
\verb+run_batch+ \\
\verb+sourcefilespath+
\color{black}

\vskip2mm
Make the \verb+compile+ and \verb+run_batch+ files executable:

\vskip2mm
\noindent\verb+~/Documents/modelJasssData$ +\color{red}\verb|chmod +x compile| \color{black}\\
\verb+~/Documents/modelJasssData$ +\color{red}\verb|chmod +x run_batch| \color{black}
\vskip2mm

\section{Compile and run the model}

To compile the model type
\vskip2mm
\noindent\verb+~/Documents/modelJasssData$ +\color{red}\verb|./compile| \color{black}
\vskip2mm

To run the model type
\vskip2mm
\noindent\verb+~/Documents/modelJasssData$ +\color{red}\verb|./run_batch| \color{black}
\vskip2mm

When the run completes, you will find the files containing the output of your simulation inside the data directory.
The data file have the \verb+z_+ prefix for ease of their identification: 

\vskip2mm
\noindent\verb+~/Documents/modelJasssData$ +\color{red}\verb|ls| \\ \color{green}
\verb+compile+ \\
\verb+configure+ \\ \color{blue}
\verb+paths.txt+ \\ \color{green}
\verb+run_batch+ \\ \color{blue}
\verb+sourcefilespath+ \\
\verb+z_consumersAskedCredit_run1.txt+ \\
\verb+z_consumersAvailableIncome_run1.txt+ \\
\verb+z_consumersConsumption_run1.txt+ \\
\verb+z_consumersEmployment_run1.txt+ \\
\verb+z_consumersObtainedCredit_run1.txt+ \\
\verb+z_consumersPossibleAdditionalConsumption_run1.txt+ \\
\verb+z_consumersProductivity_run1.txt+ \\
\verb+z_consumersRho_run1.txt+ \\
\verb+z_consumersWage_run1.txt+ \\
\verb+z_consumersWealth_run1.txt+ \\
\verb+z_datiBanca_run1.txt+ \\
\verb+z_datiFirm_run1.txt+ \\
\vskip2mm
\color{black}

Use your favorite data handling software to load the data and make analysis.


We provide hereafter the code to load the data in \verb+R+: 

\begin{verbatim}
#change the following instruction with the absolute path of the data folder 
#(only if R current working directory is different from your data folder)
dir<-"./"
#Load aggregate data
bankdata1<-read.csv(paste(dir,"z_datiBanca_run1.txt",sep=""),sep=";")
bankdata<-bankdata1[-dim(bankdata1)[1],]
firmdata1<-read.csv(paste(dir,"z_datiFirm_run1.txt",sep=""),sep=";")
firmdata<-bankdata1[-dim(bankdata1)[1],]

#Load microeconomic data
consumersWealth1<-read.csv(paste(dir,"z_consumersWealth_run1.txt",sep=""),
                                                        header=F,sep=";")
consumersWealth<-consumersWealth1[,-dim(consumersWealth1)[2]]
consumersWage1<-read.csv(paste(dir,"z_consumersWage_run1.txt",sep=""),
                                                        header=F,sep=";")
consumersWage<-consumersWage1[,-dim(consumersWage1)[2]]
consumersConsumption1<-read.csv(paste(dir,"z_consumersConsumption_run1.txt",sep=""),
                                                        header=F,sep=";")
consumersConsumption<-consumersConsumption1[,-dim(consumersConsumption1)[2]]
consumersEmployment1<-read.csv(paste(dir,"z_consumersEmployment_run1.txt",sep=""),
                                                                header=F,sep=";")
consumersEmployment<-consumersEmployment1[,-dim(consumersEmployment1)[2]]
\end{verbatim}


\section{Changing parameters}

Open the\\
\verb+/Users/coolcoder/models/modelJasss/scenario/batch_parameters.xml+ \\
file with your favorite text editor. Change the parameter values and save the file.

Run the model with the command
\vskip2mm
\noindent\verb+~/Documents/modelJasssData$ +\color{red}\verb|./run_batch| \color{black}
\vskip2mm

\subsection*{About parameters}

\subsubsection*{The random seed}
In the parameter file included in the \verb+modelJasss.zip+, the random seed is set.
This implies that each run of the model delivers exactly the same results.

 
If you need different results in different run, you have to comment the seed setting line as follows:

\color{magenta}
\begin{verbatim}
<!--  begin comment
<parameter name="randomSeed" type="constant" constant_type="int" value="1435673">
</parameter>
end comment -->
\end{verbatim}
\color{black}

\subsubsection*{Saving data}
In the parameter file included in the \verb+modelJasss.zip+, the options to save households' individual data is enabled.

In case you want to go for an aggregate analysis, you can disable these options.
This will speed up your simulations and will avoid recording huge data files especially when you set up your simulation with a large number of agents and/or you set the stopping time to a large values. 

To disable these options change \color{magenta}value=''true'' \color{black} in \color{magenta}value=''false'' \color{black} in the setting line for the parameters whose name start with \verb+saveToFile+ expression.



\end{document}
